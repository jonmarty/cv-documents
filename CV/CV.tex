%----------------------------------------------------------------------------------------
% PACKAGES AND OTHER DOCUMENT CONFIGURATIONS
%----------------------------------------------------------------------------------------

\documentclass[11pt,a4paper,sans]{moderncv} % Font sizes: 10, 11, or 12; paper sizes: a4paper, letterpaper, a5paper, legalpaper, executivepaper or landscape; font families: sans or roman

\moderncvstyle{casual} % CV theme - options include: 'casual' (default), 'classic', 'oldstyle' and 'banking'
\moderncvcolor{blue} % CV color - options include: 'blue' (default), 'orange', 'green', 'red', 'purple', 'grey' and 'black'

\usepackage{lipsum} % Used for inserting dummy 'Lorem ipsum' text into the template
\usepackage{csquotes}

\usepackage[scale=0.85]{geometry} % Reduce document margins
%\setlength{\hintscolumnwidth}{3cm} % Uncomment to change the width of the dates column
%\setlength{\makecvtitlenamewidth}{10cm} % For the 'classic' style, uncomment to adjust the width of the space allocated to your name

\renewcommand{\listitemsymbol}{~~~~} % Changes the symbol used for lists

%----------------------------------------------------------------------------------------
% NAME AND CONTACT INFORMATION SECTION
%----------------------------------------------------------------------------------------

\firstname{Jonathan} % Your first name
\familyname{Marty} % Your last name

\title{Research CV}

\begin{document}
	
\makecvtitle

\section{Contact Information}

\cvitem{\textbf{}Email:}{jonathan.n.marty@gmail.com \textit{} }
\cvitem{\textbf{}Phone:}{732-397-1450 \textit{} }
\cvitem{\textbf{}Website:}{\href{https://github.com/jonmarty}{https://github.com/jonmarty}, \href{https://www.linkedin.com/in/jonathan-m-848204137/}{https://www.linkedin.com/in/jonathan-m-848204137/} }

\section{Education}

\cvitem{2019-Present}{\textbf{BS in Computer Science, Minor in Applied Math:} School of Engineering, Columbia University, New York, NY}
\cvlistitem{GPA: 3.9, Relevant Courses: Differential Geometry (Reading course), Differentiable Manifolds, Probability Theory, Topology, Analytic Methods for PDEs, Modern Algebra I, Analysis of Algorithms I, Computational Neuroscience, Accelerated Physics I \& II}

\cvitem{2015-2019}{\textbf{High School Education:} Holmdel High School, Holmdel, NJ}
\cvlistitem{GPA: 4.6/4.0 (weighted); SAT: 1550, Verbal: 750, Math: 800; SAT Subject Tests: 800 Physics, 800 Math II; 5/5 on 8 AP Exams}

\section{Experience}

\cvitem{Spring 2022}{\textbf{Course Assistant, ECBM E4070 Computing with Brain Circuits of Model Organisms:} Course assistant in a graduate level course on the functional logic of parts of the fruit fly brain. Several brain circuits are covered, including how sensory coding in the early visual system directs movement and how odorant transduction in the early olfactory system contributes to learning and memory.  Responsibilities include grading coding-based homework assignments and projects, holding office hours, and answering student questions. (\href{http://www.bionet.ee.columbia.edu/courses/ECBM_E4070}{http://www.bionet.ee.columbia.edu/courses/ECBM\_E4070}).}

\renewcommand{\listitemsymbol}{} % Changes the symbol used for lists
\cvitem{2020-Present}{\textbf{Student Researcher, Bionet Lab, Columbia University:} A mentored individual project attempting to leverage recent advances in neuroanatomic data to push forward the functional understanding of the fruit fly (Drosophila Melanogaster) circadian clock. Currently focusing on using the frequency and phase dynamics of networks of relaxation oscillators to elucidate the transfer of relative time information in the circadian clock. Have demonstrated behaviors such as one oscillator "silencing" another and two oscillators "synchronizing" (both in phase and frequency) under mutual inhibition. Working to integrate these findings with existing connectomic and morphological data to create a computational model of the circadian clock.}
	
\cvitem{Summer 2020}{\textbf{Intern, AT\&T Labs Cyber Security Organization} Performed research aimed at detecting malicious activity in large-scale IP networks. Used temporal trends in graph structure, most notably the Triad Census of IP connections, to detect P2P (Peer to Peer) botnets, which do not share the hierarchical structure of most IP networks. Combined this analysis with several other security metrics and developed an ensemble decision tree model that detected 32\% of all suspected malicious HTTP source devices with a 25\% true positive rate.}
			
\cvitem{Summer 2018}{\textbf{Intern, Bionet Lab, Columbia University:} Performed development and research at the Columbia University Bionet lab, whose mission is to develop computation models of brain function (\href{http://www.bionet.ee.columbia.edu}{http://www.bionet.ee.columbia.edu}). Packaged Fruit Fly Brain Observatory software using Docker which led to the adoption of the research platform by other universities and is currently being used as the basis for a graduate class. Implemented a 47 layer recurrent convolutional neural network model of the fruit fly early visual system. This work was done to test the efficacy of research done in the paper "A Connectome Based Hexagonal Lattice Convolutional Network Model of the Drosophila Visual System" (\href{https://arxiv.org/abs/1806.04793}{https://arxiv.org/abs/1806.04793}). This involved building the model from scratch in Keras, implementing custom Keras layers and functionality using Tensorflow as a backend (ex. implementing hexagonal convolution), and performing statistical tests and analyses to gauge the model's performance.}

\cvitem{2017-2019}{\textbf{Adjunct Employee, Perspecta Labs, Red Bank, NJ:} Participating in DARPA RADICS (\href{https://www.darpa.mil/news-events/2015-12-14}{https://www.darpa.mil/news-events/2015-12-14}) project focused on securing our nations critical infrastructure. Developed methods to collect, clean, and normalize power grid network threat metrics. Evaluated data in unsupervised machine learning outlier detection algorithms. Developed novel method to enhance network intrusion detection techniques using neural spike train measures.}

\cvitem{Summer 2017}{\textbf{Intern, Vencore Labs, Red Bank, NJ:} Secured paid internship to participate in developing system for electric power grid control network security. Developed interface to dynamically add new threat detection metrics to power grid network security monitoring system. Contributed improvements to authentication and entitlements to support dynamic configuration. Invited to continue employment during 2017-2018 school year.}

\cvitem{2016-2018}{\textbf{Student Researcher, Holmdel High School and Vencore Labs, Red Bank, NJ:} Developed a method to reduce peak electrical power grid load from Personal Electric Vehicles (PEV) by employing distributed battery storage at charging stations. Performed large-scale power grid simulations to show that its adoption reduces average daily peak power levels by between 14\% and 24\% for typical urban and suburban grids. Presented results at IEEE conference.}

\section{Publications}

\cvitem{Under Review}{\textit{O.T. Schafer, G.J. Gutierrez, K. Li, A. Mildenhall, D. Spira, J. Marty, A.A. Lazar, M.P. Fernandez, \enquote{Connectomic Analysis of the Drosophila Lateral Neuron Clock Cells Reveals the Synaptic Basis of Functional Pacemaker Classes}, eLife Sciences Publications}}
\cvlistitem{\href{https://www.biorxiv.org/content/10.1101/2022.03.02.482743v3}{https://www.biorxiv.org/content/10.1101/2022.03.02.482743v3}}
\cvitem{Working Paper}{\textit{J. Marty, G. Di Crescenzo, \enquote{Analysis of Temporal Point Events for Network Intrusion Detection using Neural Spike Train Distance Measures}.}}
\cvitem{2018}{\textit{J. Marty, S. Pietrowicz, \enquote{Economic Incentives for Reducing Peak Power Utilization in Electric Vehicle Charging Stations}, IEEE PES Innovative Smart Grid Technologies Conference, Washington DC, Feb 19-22, 2018}}
\cvlistitem{\href{https://ieeexplore.ieee.org/document/8403377}{https://ieeexplore.ieee.org/document/8403377}}

\section{Technical Skills}

\subsection{Languages}
\cvitem{} {Python, R, Java, Matlab, Golang, LaTeX, Processing, CUDA, Javascript, C++}

\subsection{Operating Systems}
\cvitem{}{Linux, Windows, Mac, Cisco IOS}

\subsection{Tools}
\cvitem{}{Docker, Ansible, Slurm}

\subsection{Libraries}
\cvitem{} {Keras, Tensorflow, PyTorch, SciPy, networkx}

\section{Other}
\cvitem{}{\textbf{Interesting video of me:} \href{https://www.youtube.com/watch?v=CZv1PpTdIZ4}{https://www.youtube.com/watch?v=CZv1PpTdIZ4}}

\end{document}